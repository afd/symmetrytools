\chapter{\troubleshooting}\label{chapter:troubleshooting}

This section will be expanded to include a list of common problems
with the use of \topspin.  In the mean time, please email the author
with any queries.


\section{Common problems}\label{sec:troubleshooting:common}

%%%%%%%%%%%%%%%%%%%%%%%%%%%%%%%

\subsection{Missing configuration file}

\problem{There is no file named \inline{config.txt} in your working
directory when you run \topspin.}

\exampleerrormessage

\begin{lstlisting}
Error opening configuration file "config.txt", which should be
located in the directory from which you run TopSPIN.
\end{lstlisting}

\solution{Follow the instructions in \secref{example:configuration}
on how to create \inline{config.txt}.}

%%%%%%%%%%%%%%%%%%%%%%%%%%%%%%%

\subsection{Incomplete configuration file}

\problem{The file \inline{config.txt} does not contain entries for all of the
mandatory options, \inline{gap}, \inline{saucy} and \inline{common}.}

\exampleerrormessage

\begin{lstlisting}
No configuration specified for GAP.
No configuration specified for saucy.
No common directory specified.
\end{lstlisting}

\solution{Follow the instructions in \secref{example:configuration}
on how to create \inline{config.txt} with these mandatory options.}

%%%%%%%%%%%%%%%%%%%%%%%%%%%%%%%

\subsection{The \saucy\ program is not correctly installed}

\problem{You may not have correctly installed and compiled \saucy,
the graph automorphism program on which \topspin\ relies.}

\exampleerrormessage

\begin{lstlisting}
Error launching saucy with command: C:\Program Files\TopSPIN\saucy\saucy.exe -d
"C:\Program Files\TopSPIN\Common\graph.saucy"
java.io.IOException: CreateProcess: C:\Program Files\TopSPIN\saucy\saucy.exe -d
"C:\Program Files\TopSPIN\Common\graph.saucy" error=2
        at java.lang.ProcessImpl.create(Native Method)
        ... rest of Java stack trace
\end{lstlisting}

\solution{Source code for \saucy\ is included with the \topspin\
distribution, which can be downloaded via the instructions of
\secref{downloadandinstall:downloading}.  However, you need to
compile the \saucy\ source code using \gcc. To do this, follow the
instructions given in
\secref{downloadandinstall:installation:compilingsaucy}.}

%%%%%%%%%%%%%%%%%%%%%%%%%%%%%%%

\subsection{Path to \saucy\ in configuration file is wrong}

\problem{From \topspin's point of view this is the same as the
previous problem.  From your point of view there is a difference:
you may have correctly installed and compiled \saucy, but but
mistyped the path to \saucy\ in \inline{config.txt}.}

\exampleerrormessage

\begin{lstlisting}
Error launching saucy with command: C:\Program Files\TopSPIN\saucy\tsaucy.exe -d
"C:\Program Files\TopSPIN\Common\graph.saucy"
java.io.IOException: CreateProcess: C:\Program Files\TopSPIN\saucy\tsaucy.exe -d
"C:\Program Files\TopSPIN\Common\graph.saucy" error=2
        at java.lang.ProcessImpl.create(Native Method)
        ... rest of Java stack trace
\end{lstlisting}
%
Note that \topspin\ is trying to launch \inline{tsaucy.exe} rather
than \inline{saucy.exe}, due to a typo in \inline{config.txt}.

\solution{Make sure \saucy\ is correctly downloaded and compiled
(\secref{downloadandinstall:installation:compilingsaucy}), and
ensure that \inline{config.txt} contains a line for \saucy\ with
\emph{exactly} the absolute path to the tool
(\secref{example:configuration}).}

%%%%%%%%%%%%%%%%%%%%%%%%%%%%%%%

\subsection{\protect\gap\ is not correctly installed}

\problem{You may not have correctly installed \gap, the
computational group theory package on which \topspin\ relies.}

\exampleerrormessage

\begin{lstlisting}
Starting GAP with command: C:\gap4r4\bin\gap.bat -L
"C:\Program Files\TopSPIN\Common\gapworkspace" -q
Exception in thread "main"
java.io.IOException: CreateProcess: C:\gap4r4\bin\gap.bat -L
"C:\Program Files\TopSPIN\Common\gapworkspace" -q error=2
        at java.lang.ProcessImpl.create(Native Method)
        ... rest of Java stack trace
\end{lstlisting}

\solution{Download and install \gap\ from the \gap\ website.  The
URL for this website and the version of \gap\ which \topspin\
supports are given in \figref{symmextractorandtopspin:prerequisites}
(\secref{downloadandinstall:prerequisites}).}

%%%%%%%%%%%%%%%%%%%%%%%%%%%%%%%

\subsection{Path to \protect\gap\ in configuration file is wrong}

\problem{From \topspin's point of view this is the same as the
previous problem.  From your point of view there is a difference:
you may have correctly installed \gap, but mistyped the path to
\gap\ in \inline{config.txt}.}

\exampleerrormessage

\begin{lstlisting}
Starting GAP with command: C:\gap4r4\bin\tgap.bat -L
"C:\Program Files\TopSPIN\Common\gapworkspace" -q
Exception in thread "main"
java.io.IOException: CreateProcess: C:\gap4r4\bin\tgap.bat -L
"C:\Program Files\TopSPIN\Common\gapworkspace" -q error=2
        at java.lang.ProcessImpl.create(Native Method)
        ... rest of Java stack trace
\end{lstlisting}
%
Note that \topspin\ is trying to launch \inline{tgap.bat} rather
than \inline{gap.bat}, due to a typo in \inline{config.txt}.

\solution{Make sure you have correctly downloaded and installed \gap
(\secref{downloadandinstall:prerequisites}), and ensure that
\inline{config.txt} contains a line for \gap\ with \emph{exactly}
the absolute path to the tool (\secref{example:configuration}).}

%%%%%%%%%%%%%%%%%%%%%%%%%%%%%%%

\subsection{Common directory does not exist, or user does not have permissions for this directory}

\problem{The location of the \topspin\ \inline{Common} directory has
been specified incorrectly in \inline{commmon.txt}.}

\exampleerrormessage

\begin{lstlisting}
Error while trying to create file "C:\Program Files\TopSPIN\Common\graph.saucy".
Make sure that the directory C:\Program Files\TopSPIN\Common\ exists,
and that you have write permission.
\end{lstlisting}

\solution{Make sure that \inline{config.txt} contains a line of the
form \inline{common=}\emph{name}, where \emph{name} is the
\emph{absolute} path to the \inline{Common} directory provided with
the \topspin\ distribution.  Make sure this path has not been
mistyped, and that the path includes a terminating slash.}

%%%%%%%%%%%%%%%%%%%%%%%%%%%%%%%

\subsection{Path to Common directory does not have terminating slash}

\problem{The location of the \topspin\ \inline{Common} directory has
been specified without a final forward- (Linux) or back- (Windows) slash.}

\exampleerrormessage

\begin{lstlisting}

\end{lstlisting}

\solution{Make sure that \inline{config.txt} contains a line of the
form \inline{common=}\emph{name}, where \emph{name} is the
\emph{absolute} path to the \inline{Common} directory provided with
the \topspin\ distribution.  Make sure this path has not been
mistyped, and that the path includes a terminating slash.}

%%%%%%%%%%%%%%%%%%%%%%%%%%%%%%%

\subsection{Missing or corrupt \protect\gap\ workspace}

\problem{The file \inline{gapworkspace} in the \inline{Common}
directory is either missing or corrupted.}

\exampleerrormessage

\begin{lstlisting}
Starting GAP with command: C:\gap4r4\bin\gap.bat -L
"C:\Program Files\TopSPIN\Common\gapworkspace" -q
Error -- bad GAP workspace specified in configuration file.

GAP produced errors:
====================
gap: Press <Enter> to end program
End of GAP errors
\end{lstlisting}

\solution{Check that:
%
\begin{enumerate}
\item You have followed the instructions in \secref{downloadandinstall:gapworkspace} on creating a \gap\ workspace
\item This workspace has been successfully saved in the \inline{Common} directory
\item You have not changed the version of \gap\ you are using since creating the workspace
\item The workspace was created exactly the same machine, with the same operating system,
on which you are running \topspin.
\end{enumerate}
}

%%%%%%%%%%%%%%%%%%%%%%%%%%%%%%%

\subsection{\protect\gap\ executable specified in configuration file, instead of shell script/batch file}

\problem{You have specified the path to the \gap\ executable as the \inline{gap} option in \inline{config.txt}.}

\exampleerrormessage

\begin{lstlisting}
GAP produced errors:
====================
Error, the library file 'system.g' must exist and be readable called from
CallFuncList( func, arg ); called from
RereadLib( "system.g" ); called from
<function>( <arguments> ) called from read-eval-loop
Entering break read-eval-print loop ...
you can 'quit;' to quit to outer loop, or
you can 'return;' to continue
brk>
\end{lstlisting}

\solution{Open \inline{config.txt} and change the \inline{gap} option to refer to the shell script
(\inline{gap.sh}) or batch file (\inline{gap.bat}).}

%%%%%%%%%%%%%%%%%%%%%%%%%%%%%%%

\subsection{\protect\spin\ is not correctly installed}

\problem{You have not correctly installed \spin\ in such a way that
the tool can be invoked by simply typing \inline{spin}, as discussed
in \secref{downloadandinstall:prerequisites}.}

\exampleerrormessage

\begin{lstlisting}
An error occurred while constructing the "sympan" files.
java.io.IOException: CreateProcess: spin -a loadbalancer.p error=2
        at java.lang.ProcessImpl.create(Native Method)
        ... rest of Java stack trace
\end{lstlisting}

\solution{Make sure \spin\ is correctly installed, and that the
directory containing the \spin\ executable is part of your
\emph{path} environment variable.  Perhaps you have
renamed \inline{spin} to \inline{spin-linux} or \inline{spin516};
maybe you do not have execute permission for \spin, or perhaps you
have another application called \inline{spin} installed on your
machine. Resolve these issues so that \inline{spin} is all that
needs to be typed to launch the \spin\ tool.}

%%%%%%%%%%%%%%%%%%%%%%%%%%%%%%%

\subsection{C preprocessor, \protect\inline{cpp}, unavailable}

\problem{\topspin\ uses the \inline{cpp} program to expand \inline{#define} and \inline{#include}
macros before processing a specification.  If \inline{cpp} is not available then \topspin\ can
still be invoked, but cannot handle specifications which include these directives.}

\exampleerrormessage

\begin{lstlisting}
C preprocessor (cpp) not available - TopSPIN will not work correctly on files which use
#define or #include.
src.promela.lexer.LexerException: [1,1] Unknown token: #
    ... rest of Java stack trace
\end{lstlisting}

\solution{Install the \inline{cpp} program on your system.  Linux distributions provide
\inline{cpp} as standard; under Windows the \inline{cpp} program is provided as part of
Cygwin.  Make sure \inline{cpp} is in your path.}

%%%%%%%%%%%%%%%%%%%%%%%%%%%%%%%

\subsection{Typechecking error: problem with array index}

\problem{By default, \topspin\ performs strict typechecking and only allows arrays to be accessed using \texttt{byte} or \texttt{pid} types.}

\exampleerrormessage

\begin{lstlisting}
Error at line 8: Type "int" cannot be used as an array index, it is not a subtype of "byte".
\end{lstlisting}

\solution{You can use the \texttt{-relaxedarrayindexing} option to make this error go away (see \secref{overview:commandline:relaxedarray}), or refactor your specification so
that arrays are indeed only indexed by \texttt{byte} or \texttt{pid} expressions: this may make the associated state-vector smaller.}

%%%%%%%%%%%%%%%%%%%%%%%%%%%%%%%

\subsection{Typechecking error: problem with assignment to numeric variable}

\problem{By default, \topspin\ performs strict typechecking and only allows a variable to be assigned a value of a ``smaller'' type, \eg assignment of
\texttt{byte} to \texttt{short} is OK, but not the other way round.  Sometimes this is too restrictive.}

\exampleerrormessage

\begin{lstlisting}
Error at line 8: invalid assignment -- Type "short" occurs in a context where it is required to be a subtype of "byte".
\end{lstlisting}

\solution{You can use the \texttt{-relaxedassignment} option to make this error go away (see \secref{overview:commandline:relaxedassign}),
or refactor your specification so that assignments to numeric variables are always from expressions with a smaller type: this may make the
associated state-vector more compact.}

%%%%%%%%%%%%%%%%%%%%%%%%%%%%%%%


%\subsection{}
%
%\problem{}
%
%\exampleerrormessage
%
%\begin{lstlisting}
%\end{lstlisting}
%
%\solution{}

\section{Limitations of \protect\topspin}
%
\topspin\ currently supports the verification of \emph{safety}
properties, expressed using assertions or monitor processes.  The
tool does not yet support symmetry-reduced \ltl\ model checking
using never claims.
%
\section{Reporting bugs in \protect\topspin}\label{sec:troubleshooting:reportingbugs}
%
If you have a problem using \topspin, first check to see if your
problem is covered by one of the \emph{common problems} in
\secref{troubleshooting:common}.  If this is not the case, then
please take the time to report your bug to the \topspin\ developers,
so that it can be immediately documented and ultimately fixed, for
the benefit of you and other \topspin\ users.

Bug reports should be submitted using the contact details in
\secref{troubleshooting:gettingintouch}.  Please provide the
following when reporting a bug:

\begin{itemize}
\item A description of the problem
\item A test-case (example Promela specification and \inline{config.txt} file) which exposes the problem
\item The \topspin\ version you are using, the approximate date on which you downloaded \topspin
\item Details as to whether you are using a pre-compiled version of \topspin, or a version you have compiled from source
\item Any ideas you have as to what may have gone wrong!
\end{itemize}

Please note that bugs do not have to relate to the correctness of
\topspin\ -- a valuable bug report could be in response to a
misleading error message generated by \topspin, where there
\emph{is} something wrong with the input specification, but the
problem is not what the error message indicates.  Please feel free
to also submit suggestions for features to be added to \topspin.

\section{Reporting bugs in this manual}

Please also get in touch (\secref{troubleshooting:gettingintouch})
if you find typos, inconsistencies or ambiguities in this manual --
this kind of feedback is extremely valuable.

\section{Getting in touch}\label{sec:troubleshooting:gettingintouch}

All correspondence related to \topspin\ should be by email, to
Alastair Donaldson: \inline{ally@codeplay.com}.
