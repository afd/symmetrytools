\chapter{\buildingfromsource}\label{chapter:compilingfromsource}


\section{Downloading \protect\topspin\ source
code}\label{sec:compilingfromsource:downloading}

The \topspin\ source code is stored in a Subversion repository,
hosted by the Department of Computing Science at the University of
Glasgow at the following URL:
%
\begin{lstlisting}
https://ouen.dcs.gla.ac.uk/repos/symmetry/
\end{lstlisting}
%
Use subversion to check out the \topspin\ source code to an
appropriate location:
%
\begin{lstlisting}
C:\prog>svn checkout https://ouen.dcs.gla.ac.uk/repos/symmetry/trunk TopSPINSource
A TopSPINSource/TestModels
A TopSPINSource/TestModels/EtchTesting
A TopSPINSource/TestModels/EtchTesting/ParsePassTests
A TopSPINSource/TestModels/EtchTesting/ParsePassTests/peterson
A TopSPINSource/TestModels/EtchTesting/ParsePassTests/hello
A TopSPINSource/TestModels/EtchTesting/ParsePassTests/pathfinder
A TopSPINSource/TestModels/EtchTesting/ParsePassTests/snoopy
A TopSPINSource/TestModels/EtchTesting/ParsePassTests/mobile1
A TopSPINSource/TestModels/EtchTesting/ParsePassTests/mobile2
A TopSPINSource/TestModels/EtchTesting/ParsePassTests/pftp
A TopSPINSource/TestModels/EtchTesting/ParsePassTests/leader
A TopSPINSource/TestModels/EtchTesting/ParsePassTests/loops
...
A TopSPINSource/Common/Minimising.gap
A TopSPINSource/Common/parallel_symmetry_cell_ppu.c
A TopSPINSource/Common/Verify.gap
A TopSPINSource/Common/WorkspaceGenerator.gap
A TopSPINSource/Common/parallel_symmetry_cell_spu.c
A TopSPINSource/Common/parallel_symmetry_cell_ppu.h
A TopSPINSource/symmextractor_common_config.txt
Checked out revision 75.
\end{lstlisting}

\section{Generating the Promela parser}\label{sec:compilingfromsource:generating}

\topspin\ is based on a Promela parser which is constructed using
the SableCC parser generator.  Download SableCC version
\sableccversion\ from the Sable
website\footnote{\texttt{\sableccwebsite}}, and generate the parser
as follows (adapting the command according to where you have
installed SableCC):
%
\begin{lstlisting}
java -jar C:\sablecc-3.2\lib\sablecc.jar promela.grammar
\end{lstlisting}
%
This command should result in output similar to the following:
%
\begin{lstlisting}
C:\prog\TopSPINSource>java -jar C:\sablecc-3.2\lib\sablecc.jar
promela.grammar

SableCC version 3.2 Copyright (C) 1997-2003 Etienne M. Gagnon <etienne.gagnon@uqam.ca>
and others.  All rights reserved.

This software comes with ABSOLUTELY NO WARRANTY.  This is free software,
and you are welcome to redistribute it under certain conditions.

Type 'sablecc -license' to view the complete copyright notice and license.

 -- Generating parser for promela.grammar in C:\prog\TopSPINSource
Adding productions and alternative of section AST.
Verifying identifiers.
Verifying ast identifiers.
Adding empty productions and empty alternative transformation if necessary.
Adding productions and alternative transformation if necessary.
computing alternative symbol table identifiers.
Verifying production transform identifiers.
Verifying ast alternatives transform identifiers.
Generating token classes.
Generating production classes.
Generating alternative classes.
Generating analysis classes.
Generating utility classes.
Generating the lexer.
 State: INITIAL
 - Constructing NFA.
..............................................................................
..............................................................................
 - Constructing DFA.
..............................................................................
..............................................................................
 - resolving ACCEPT states.
Generating the parser.
..............................................................................
..............................................................................
\end{lstlisting}

Due to a bug in SableCC, it is then necessary to apply a patch file to the generated parser.  Do this by executing
the following command (you will need to have PERL installed: if you don't, you could easily re-write the
PERL script in your favourite language!).

\begin{lstlisting}
C:\prog\TopSPINSource>perl apply_patch.pl
\end{lstlisting}

\section{Compiling and creating a jar}\label{sec:compilingfromsource:compiling}

You have now downloaded and generated all the Java source code for
\topspin, and can proceed to compile this source code.  The
\inline{javac} compiler and \inline{jar} utility are required to
compile the source and create an executable jar file respectively.
The source code is all contained in directories within \inline{src}.

\subsection{Compiling}

There are two options for compilation:

\begin{enumerate}

\item Create an Eclipse project from the \topspin\ source code, in which case
Eclipse will automatically compile the code.

\item Use the \inline{make} utility and the supplied \inline{Makefile} to compile
the source code, by typing the command \inline{make classes} in the \topspin\
root directory. The provided \inline{Makefile} uses the \inline{sed}
and \inline{find} utilities, which are provided as standard with
Linux, and are available on Windows via Cygwin.

\end{enumerate}
%
Compiling the source code using the \inline{Makefile} gives output
along the following lines:
%
\begin{lstlisting}
C:\prog\TopSPINSource>make classes
javac src/etch/checker/Check.java
Note: .\src\promela\parser\Parser.java uses unchecked or unsafe operations.
Note: Recompile with -Xlint:unchecked for details.
javac src/etch/checker/CheckerTest.java
javac src/etch/testing/EtchTestCase.java
javac src/etch/testing/EtchTester.java
javac src/group/Group.java
javac src/promela/analysis/ReversedDepthFirstAdapter.java
javac src/symmextractor/InlineReplacer.java
javac src/symmextractor/SymmExtractor.java
javac src/symmextractor/testing/SymmExtractorFailTestOutcome.java
javac src/symmextractor/testing/SymmExtractorRunTestOutcome.java
javac src/symmextractor/testing/SymmExtractorTestCase.java
javac src/symmextractor/testing/SymmExtractorTester.java
javac src/symmreducer/testing/ModelCheckingResult.java
javac src/symmreducer/testing/SymmReducerFailTestOutcome.java
javac src/symmreducer/testing/SymmReducerTestCase.java
javac src/symmreducer/testing/SymmReducerTester.java
javac src/testing/RunAllTests.java
\end{lstlisting}
%
Note the compile warnings generated for \inline{Parser.java}: these
are due to code generated by SableCC.  All other files should
compile without warnings.
%
\subsection{Creating a jar file}
%
If you are using Eclipse then you can create an executable jar file
for \topspin\ by exporting your \topspin\ project as a jar, and
selecting \inline{src.TopSpin} as the \emph{main} class.

If using the supplied \inline{Makefile}, typing \inline{make jars}
will produce two jar files:
%
\begin{lstlisting}
C:\prog\TopSPINSource>make jars

jar cmf manifest.txt TopSPIN.jar src/etch/checker/Check.class src/etch/checker/Checker.class
... <many .class files> ...
src/utilities/Strategy.class src/utilities/StringHelper.class src/promela/lexer/lexer.dat
src/promela/parser/parser.dat && echo "TopSPIN.jar built successfully."
TopSPIN.jar built successfully.

jar cmf tests_manifest.txt TopSPINTests.jar src/etch/checker/Check.class
src/etch/checker/Checker.class
... <many .class files> ...
src/utilities/Strategy.class src/utilities/StringHelper.class src/promela/lexer/lexer.dat
src/promela/parser/parser.dat && echo "TopSPINTests.jar built successfully."
TopSPINTests.jar built successfully.
\end{lstlisting}
%
Note that you can skip the \inline{make classes} step, and type
\inline{make jars} to both compile the Java files and produce jar
files.  To delete all jar and class files, use \inline{make clean}.

\section{Try your compiled version on an example}

Now that you have successfully compiled a jar file from the
\topspin\ source, you are effectively in the same position as a user
who has downloaded the \topspin\ jar file using the instructions
given in \secref{downloadandinstall:downloading}.  Of course, you
have the advantage of being able to modify \topspin\ to suit your
purposes!

The next steps involve compiling \saucy, creating a \gap\ workspace,
and testing \topspin\ on a simple example.  Therefore you should
work your way through \chapsref{downloadandinstall}{example}, using
your compiled jar file in place of the downloaded jar file.

\section{Acceptance Tests}\label{sec:compilingfromsource:testing}

The \topspin\ source checkout includes a reasonably large set of
acceptance tests.  It is highly recommended that you run these
tests using the instructions below before commencing any development
work on \topspin\ -- passing the acceptance tests confirms that
you are starting from a stable base.  The acceptance tests can
also be run regularly during development, to ensure that the
addition of new features to \topspin\ does not adversely affect
the tool's existing features.

\vspace{2mm}
\noindent{\bf Important } Before running the acceptance tests, make sure that
the \gcc\ tool is installed in such a way that it can be invoked by
name from a command prompt, i.e. so that typing \inline{gcc} will
launch \gcc. To check this, try typing \inline{gcc} from a fresh
prompt.  You should see something like:
\vspace{2mm}
%
\begin{lstlisting}
C:\>gcc
gcc: no input files
\end{lstlisting}

\subsection{Setting up a configuration file for testing}

The \topspin\ test suite uses a called \inline{symmextractor_common_config.txt}
for some configuration options which apply to most tests.  Open
this file, and edit the \inline{gap} line to use the appropriate command
for your setup.  You should not have to change the \inline{Common} or
\inline{saucy} lines: the source code checkout is organised so that the
\inline{saucy} and \inline{Common} directories are sub-directories of
the directory in which \inline{symmextractor_common_config.txt} is
contained.

\subsection{Running the tests}

You can now run the acceptance
tests via the \inline{TopSPINTests.jar} file:
%
\begin{lstlisting}
java -ea -jar TopSPINTests.jar 2> temp.txt
\end{lstlisting}
%
The \inline{-ea} argument switches on assertion checking, which is useful for
finding potential problems with \topspin.  The final part of the command,
\inline{2> temp.txt}, pipes error messages generated during testing to the
file \inline{temp.txt}.  This is useful since many of the tests are
\emph{fail} tests, which check that the \topspin\ typechecker correctly
rejects Promela specifications which are not suitable for processing by
\topspin.  Redirecting these error messages to a text file means that
the screen is not cluttered with error messages.

You should see something like this when you run the tests:
%
\begin{lstlisting}
ETCH TESTS
==========
   [PASS] expected and actual: BadlyTyped, file: TestModels/EtchTesting/FailTests/failrec...
   [PASS] expected and actual: WellTyped, file: TestModels/EtchTesting/PassTests/testtele...
   [PASS] expected and actual: WellTyped, file: TestModels/EtchTesting/PassTests/test_ex_...
   [PASS] expected and actual: WellTyped, file: TestModels/EtchTesting/PassTests/testtele...
   [PASS] expected and actual: BadlyTyped, file: TestModels/EtchTesting/FailTests/faildup...
   [PASS] expected and actual: ParsePass, file: TestModels/EtchTesting/ParsePassTests/pet...
   ...


SYMMEXTRACTOR TESTS
===================
   [PASS] expected and actual: (well typed, group size = 72, coset search: no), file: Tes...
   [PASS] expected and actual: (well typed, group size = 1296, coset search: no), file: T...
   [PASS] expected and actual: BreaksRestrictions, file: TestModels/SymmExtractorTests/Ba...
   [PASS] expected and actual: (well typed, group size = 3628800, coset search: no), file...
   [PASS] expected and actual: BreaksRestrictions, file: TestModels/SymmExtractorTests/Ba...
   ...


SYMMREDUCER TESTS
=================
   [PASS] expected and actual: ((well typed, group size = 120, coset search: no), no. sta...
   [PASS] expected and actual: ((well typed, group size = 3628800, coset search: no), no....
   [PASS] expected and actual: ((well typed, group size = 6, coset search: no), no. state...
   [PASS] expected and actual: ((well typed, group size = 5040, coset search: no), no. st...
   [PASS] expected and actual: ((well typed, group size = 720, coset search: no), no. sta...
   ...


Summary:
   238 passes
   0 fails


Acceptance tests passed - you may commit your changes!
\end{lstlisting}
%
There are in the order of 250 test cases.  These are divided into
\etch\ tests, which test the typechecking component of \topspin;
\symmextractor\ tests, which check the symmetry detection
capabilities of the tool, and \symmreducer\
tests, which perform symmetry reduction on Promela examples, checking
that verification results for these examples are as expected.
The \etch\ tests run very quickly, the \symmextractor\ tests run
at a moderate speed, and the \symmreducer\ tests run fairly slowly.
Testing takes approximately 10 minutes on an average PC.

\subsection{Problems running the tests}

If you have compiled a fresh checkout of \topspin\ then all of the acceptance
tests should pass, since their passing is a condition for committing
changes to the source code repository.  Therefore, if you have any problems
running the tests this is likely due to a problem with the way you have
set up \gap, \saucy, \spin, or \gcc.  Have a look at \secref{troubleshooting:common}
which details common problems encountered when using \topspin.  If test
cases continue to fail then please send a bug report to the \topspin\
development team: see \secref{troubleshooting:reportingbugs} for details.
