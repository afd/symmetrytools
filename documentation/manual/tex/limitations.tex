\chapter{\limitations}\label{chapter:limitations}

Before using \topspin\ for advanced verification it is important to understand
what can and cannot be done with the tool.
\topspin\ places various restrictions on the form a Promela specification may have.
Also, certain \spin\ options are not (yet) compatible with symmetry reduction.

This section briefly described the limitations of \topspin.  Limitations marked
with an asterisk could be removed with relative ease: they just haven't been dealt
with yet.  If you're working with \spin\ and \topspin\ and are being hindered by
these restrictions then please get in tough (\seesec{troubleshooting:gettingintouch}) and efforts will be made
to add appropriate features to \topspin.

\section{Process instantiation and dynamic process creation}

\topspin\ requires all processes to be instantiated using
\inline{run} statements within an \inline{init} process.  The form of the \inline{init}
process must be:

\begin{lstlisting}
init {
  atomic {
     run ...;
     run ...;
     ...
     run ...;
     <other statements>
  }
}
\end{lstlisting}

\noindent \ie all process must be started atomically, and must come before any other statements in the \inline{init}
process (\eg initialisation code), which must also appear within the \inline{atomic} block.

In particular, \topspin\ does \emph{not} support process instantiation using the \inline{active} keyword,
or via \inline{run} statements outside the \inline{init} process.


%
\begin{figure}
\begin{scriptsize}
\begin{verbatim}
proctype P() {

  /* body */

}

proctype Q() {
   ...

   run P();

   ...
}
\end{verbatim}
\end{scriptsize}
\caption{Skeleton Promela specification with dynamic process
creation.}\label{fig:limitations:originaldynamic}
\end{figure}
%
\begin{figure}[t]
\begin{scriptsize}
\begin{verbatim}
chan wakeup_P_1 = [0] of {bit};
chan wakeup_P_2 = [0] of {bit};
chan wakeup_P_3 = [0] of {bit}

proctype P(chan start) {

sleep:
   start?1;

   /* body */

   goto sleep
}

proctype Q() {
   ...

   if :: wakeup_P_1!1
      :: wakeup_P_2!1
      :: wakeup_P_3!1
   fi;

   ...
}

init {
   atomic {
      /* original `run' statements */

      run P(wakeup_P_1);
      run P(wakeup_P_2);
      run P(wakeup_P_3)
   }
}
\end{verbatim}
\end{scriptsize}
\caption{Re-modelled Promela specification without dynamic process
creation.}\label{fig:limitations:remodelleddynamic}
\end{figure}


If a specification relies on dynamic process
creation then it may be possible to re-model the processes as shown
in Figures~\ref{fig:limitations:originaldynamic}
and~\ref{fig:limitations:remodelleddynamic}.
\figref{limitations:originaldynamic} shows a
specification which instantiates copies of proctype $P$ dynamically.
Assuming that 3 is an upper bound for the number of instances of $P$
which should be running at any time,
\figref{limitations:remodelleddynamic} shows an
alternative way of expressing the specification.  The proctype $P$
now includes a channel parameter, and an instance of $P$ waits until
it can receive on this channel before executing its body.  Its body
is identical to the original, except that it includes a final
\inline{goto} statement after which it returns to its initial
configuration.\footnote{This \inline{goto} statement should really
be part of an atomic block which also resets any local variables of
the proctype to their initial values.}  The \inline{init} process
instantiates three copies of $P$, each with a distinct synchronous
channel.  Instead of instantiating a copy of $P$, the proctype $Q$
now offers the literal value 1 to all channels on which instances of
$P$ may be listening.  The example of
Figures~\ref{fig:limitations:originaldynamic}
and~\ref{fig:limitations:remodelleddynamic} can be
adapted to handle multiple process types, with any fixed upper bound
for each process type.



\section{Process termination}\label{sec:limitations:termination}

Technically, process termination destroys symmetry in a Promela specification.  \spin\ only allows processes
to terminate in \emph{reverse} order.  So, if five identical, terminating \texttt{client} proctypes are launched simultaneously
with process identifiers in the range $\{1,2,3,4,5\}$, the processes will terminate one-by-one in a fixed order: 5, 4, 3, 2, 1.
This ordering on process identifiers breaks symmetry between the processes.

Furthermore, the way symmetry reduction is implemented in \topspin\ is not strictly compatible with process termination.  After the
initial state, \topspin\ assumes a fixed set of running processes.  If a process dies, \topspin\ may try to exchange this
processes's state with that of another process when computing symmetry representatives, and this can (very occasionally) lead
to verification-time errors (\seesec{troubleshooting:common:termination}).

If you want to work with terminating processes, you can insert a dummy termination state at the end of each proctype using
a line like the following:

\begin{lstlisting}
end_ok: false
\end{lstlisting}

The \inline{end} label ensures that \spin\ will not complain about instances of the proctype blocking at this label,
and the \inline{false} statement ensures that proctype instances will never truly terminate.

\section{The \protect\inline{_pid} variable should be used}

For symmetry to be detected, it is important for proctypes to use
their built-in \inline{_pid} variable rather than a user-defined
process identifier.  This is illustrated in
Figures~\ref{fig:limitations:originalpid}
and~\ref{fig:limitations:remodelledpid}. Processes in
\figref{limitations:originalpid} are parameterised by a
{\it byte} identifier, which they use to index the \texttt{st}
array. SymmExtractor is not yet sophisticated enough to work out the
correspondence between the \texttt{id} parameter and the built-in
identifier for each process.  However, the specification can be
converted into a form which SymmExtractor can handle, as shown in
\figref{limitations:remodelledpid}. The disadvantage
here is that position 0 of the array \texttt{st} is un-used, meaning
that an array of size three rather than two is required, increasing
the state-vector size by one byte. On the other hand, eliminating
the \texttt{id} variables reduces the state-vector by two bytes, so
the re-modelling works well for this example.

\section{Restrictions on channels*}
For simplicity, \topspin\ does not allow channel initialisers to be associated
with channels which are declared locally to a proctype, and does not currently
support arrays of channels.

\begin{figure}
\begin{scriptsize}
\begin{verbatim}
mtype = {N,T,C}
mtype st[2]=N

proctype user(byte id) {
  do
    :: d_step { st[id]==N -> st[id]=T }
    :: d_step { st[id]==T && st[0]!=C && st[1]!=C -> st[id]=C }
    :: d_step { st[id]==C -> st[id]=N }
  od
}

init {
  atomic {
    run user(0);
    run user(1);
  }
}
\end{verbatim}
\end{scriptsize}
\caption{Promela specification which uses user-defined process
identifiers.}\label{fig:limitations:originalpid}
\end{figure}

\begin{figure}
\begin{scriptsize}
\begin{verbatim}
mtype = {N,T,C} mtype st[3]=N;

proctype user() {
  do
    :: d_step { st[_pid]==N -> st[_pid]=T }
    :: d_step { st[_pid]==T && st[1]!=C && st[2]!=C -> st[_pid]=C }
    :: d_step { st[_pid]==C -> st[_pid]=N }
  od
}

init {
  atomic {
    run user();
    run user();
  }
}
\end{verbatim}
\end{scriptsize}
\caption{Re-modelled specification which uses the \protect\inline{_pid}
variable.}\label{fig:limitations:remodelledpid}
\end{figure}

\section{Never claims, trace/notrace constructs, \protect\inline{accept} and \protect\inline{progress} labels}

These are method for specifying liveness properties in a Promela model.  \topspin\ is not
yet compatible with verification of liveness properties, so you will get an error if you try
to apply \topspin\ to a specification containing one of these constructs.

Symmetry \emph{detection} is still possible with liveness verification constructs.  It is OK
to apply \topspin to a specification containing a liveness construct with the \texttt{-detect}
option (\seesec{troubleshooting:gettingintouch}).  The tool will tell you any symmetries it detects; it's just not yet
possible to exploit these symmetries when model checking.

\section{Exclusive send/recieve (\protect\inline{xs}/\protect\inline{xr}) channel assertions}

This is not actually a restriction, rather a warning for users wishing to use these keywords.

Promela includes keywords
\inline{xr} and \inline{xs},
which stand for {\it exclusive receive} and {\it exclusive send}
respectively. A process can include a declaration `\inline{xr}
\emph{name}', where `\emph{name}' is the name of a previously declared
channel, to indicate that only this process can receive messages on
the channel. The \inline{xs} keyword is used similarly.  Providing
\spin\ with this information can lead to more efficient
partial-order reduction. It is not
possible to check, statically, whether \inline{xs} and \inline{xr}
are used correctly, but incorrect uses are flagged by \spin\ during
verification.  These keywords do not affect the presence of symmetry
in the model associated with a specification, so are no problem for
symmetry detection.

However, there is a potential problem with exploiting
\inline{xs}/\inline{xr} information in conjunction with symmetry
reduction, described in Appendix C.1.1 of Donaldson's thesis.  In practice, the
problem with these keywords is unlikely to cause problems: there
is a slim chance that if the keywords are used incorrectly, symmetry reduction
will prevent this from being flagged up during verification.  Therefore, \topspin\
\emph{does} allow these features to be used, but it is the user's responsibility to
use them with care.



\section{Unsigned data type*}

Promela supports an \inline{unsigned}
numeric type. A declaration of the form \inline{unsigned} $x$ : $y$
declares an integer variable $x$ which takes non-negative values
which can be represented using $y$ bits.  Clearly the use of this
data type will have no effect on our symmetry detection/reduction
techniques. However, \topspin\ is integrated with an enhanced
Promela type checker which does not currently support the \inline{unsigned}
data type.  A temporary fix for this omission is to replace each occurrence of the
\inline{unsigned} keyword with one of the other numeric types during
symmetry detection.

\section{Sorted send, random receive (\protect\inline{!!} and \protect\inline{??} operators)}

Promela provides alternative channel operators: \inline{!!} (sorted
send) and \inline{??} (random receive).
Sending data on a buffered channel using \inline{!!} causes messages to be queued on the
channel in sorted order. Messages can be retrieved from the buffer
in a random order using \inline{??}. These operators provide a
useful alternative to FIFO channel semantics. They also aid
state-space reduction: storing channel contents in a sorted manner
can be seen as a form of state canonicalisation. However, storing
\inline{pid} messages in a sorted queue imposes an ordering on the set
of process identifiers.  It is not immediately clear whether this
ordering has an effect on symmetry.

For the time being, \topspin\ allows the \inline{!!} and \inline{??} operators to be
used, but they should be used with care.  If you get weird symmetry reduction
results using these operators then please get in touch (\secref{troubleshooting:gettingintouch}).

\section{Embedded C code}

Recent versions of \spin\ allow C code to be embedded in a Promela
specification, and certain variables from the C part of the
specification to be included in the \spin\ state-vector.  Automatic
symmetry detection for this mix of C and Promela is beyond the scope
of \topspin\ at present.


\section{Breadth-first search*}

\topspin\ has not been tested with breadth-first search, and will give an
error message if \texttt{sympan.c} is compiled with \texttt{-DBFS}.
