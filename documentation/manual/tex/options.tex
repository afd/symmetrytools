\chapter{\overviewofoptions}\label{chapter:overview}

This chapter will be expanded into a complete reference for the
various \topspin\ options.

%\section{The \protect\etch\ typechecker}\label{sec:overview:etch}
%
%\section{SymmExtractor}\label{sec:overview:symmextractor}
%
%\section{\protect\topspin\ strategies and options}\label{sec:overview:strategiesandoptions}


%
%%
%\section{Running \protect\topspin}
%%
%To run \topspin, navigate to the folder in which you would like your
%executable verifier to be generated (as you would when using \spin\
%as standard).  Copy the \texttt{config} file to this directory. Then
%type
%\begin{verbatim}
%java -jar topspinfolder/TopSPIN.jar detect_opt reduce_opt model
%\end{verbatim}
%Here \texttt{model} is the name of the Promela specification you
%wish to check, \texttt{detect\_opt} is a symmetry detection option
%(see Section~\ref{sec:running:detection}), and
%$\texttt{reduce\_opt}$ is a symmetry reduction option (see
%Section~\ref{sec:running:reduction}). If neither option is
%specified, the Promela file is just type-checked. If only a symmetry
%detection option is given, the detected symmetries are displayed but
%no verifier is generated.
%
%If both symmetry reduction options are specified then an executable
%verifier, \texttt{sympan.c}, is generated.  This can be compiled and
%executed using the same method that would be used with the
%\texttt{pan.c} file generated by \spin, except that the file
%\texttt{group.o} must appear in the compilation command.  For
%example, one could compile \texttt{sympan.c} as follows:
%%
%\begin{verbatim}
%gcc -w -o sympan -D_POSIX_SOURCE -MEMLIM=256 -DSAFETY -DNOCLAIM
%-DNOFAIR sympan.c group.o
%\end{verbatim}
%%
%\subsection{Symmetry Detection Options}\label{sec:running:detection}
%%
%\topspin\ provides two techniques for automatic symmetry detection
%-- {\it static channel diagram} or {\it identifier adjacency
%diagram} analysis.  Additionally, \topspin\ allows generators for a
%symmetry group to be specified manually.  The following options
%specify which symmetry detection method should be used, and should
%be given as the first parameter to \topspin:\ \\
%
%\noindent{\bf -scd} With this option, symmetry detection is via
%static channel diagram analysis.  Details of this technique in
%general can be found in \cite{domica}, with \spin-specific details
%in \cite{domi}. Static channel diagram analysis works well when
%fixed communication links
%are passed to processes as parameters on instantiation.\ \\
%
%\noindent{\bf -iad} {\it Identifier adjacency diagrams} are a
%(currently unpublished) alternative to static channel diagrams, and
%are suitable for symmetry detection when the communication structure
%of a model is encoded in the Promela specification via a lookup
%macro. For example.......   Occurrences of literal {\it pid} values
%or channel names are used to build a directed graph, from which
%model automorphisms are derived using the computational group
%theoretic
%approach of \cite{domi}.\ \\
%
%\noindent{\bf -manual $<$filename$>$}
%
%\noindent If a specification is know to exhibit component symmetry,
%but this symmetry is not detected using the above approaches, it is
%possible to manually specify generators for a symmetry group.  The
%file specified by {\bf $<$filename$>$} should contain one line per
%generator, and each generator should be a product of permutations of
%global channel names and literal {\it pid} values.
%%
%\subsection{Symmetry Reduction Options}\label{sec:running:reduction}
%
%\noindent{\bf -enumerate} The {\it enumerate} symmetry reduction
%option computes orbit representatives by exhaustively enumerating
%images of a state under a group $G$.  Thus the time complexity of
%computing a representative is proportional to $|G|$, which may be as
%large as $|n!|$ where $n$ is the number of system components.
%Enumeration is memory optimal and is suitable when $G$ is small, or
%for purposes of
%comparison with other strategies.\ \\
%
%\noindent{\bf -fast} With the {\it fast} option, a symmetry
%reduction strategy will be chosen for the group $G$ based on its
%structure, as described in \cite{domi3}.  If $G$ is small, the {\it
%enumerate} option may be used.  If $G$ is unclassifiable, the
%approximate {\it hillclimbing} option (below) will be used.
%Otherwise a strategy based on minimising sets for $G$ will be
%applied, resulting in fast representative computation.  Memory
%optimality is not guaranteed if components in the specification hold
%references to one another, but in practice a close-to-optimal factor
%of reduction is usually
%achieved.\ \\
%
%\noindent{\bf -hillclimbing}  This option means that representatives
%will be computed using an approximate strategy described in
%\cite{domi3}
%based on hillclimbing local search.\ \\
%
%\noindent{\bf -segment}  This is an experimental option described in
%\cite{domi4} which combines features of the {\it fast} and {\it
%enumerate} options to provide memory optimal verification using an
%approach which, for groups with identifiable structure, is less
%na\"ive than pure enumeration.  To use this option, you must copy
%your executable \texttt{sympan} file into
%\texttt{topspinfolder/Common}, navigate there, launch \gap, and type
%%
%\begin{verbatim}
%Read("Verify.gap");
%
%Verify("sympan",[]);
%\end{verbatim}
%%
%Currently it is not possible to pass command line arguments to
%\texttt{sympan} this way, so you must make modifications directly to
%\texttt{sympan.c} before compiling it.  The most common command line
%argument is one to change the maximum search depth.  To do this via
%\texttt{sympan.c}, open the file and find the line where the
%variable \texttt{maxdepth} is initialised (to 10000 by default).
%Change the initial value to a more appropriate one.
%
%
