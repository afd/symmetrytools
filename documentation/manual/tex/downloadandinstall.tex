\chapter{\downloadingandinstalling}\label{chapter:downloadandinstall}

\section{Prerequisites}\label{sec:downloadandinstall:prerequisites}
%
\topspin\ is written in Java and \gap, interfaces with the \gap\ and
\spin\ packages, and produces C code which must then be compiled.
\figref{symmextractorandtopspin:prerequisites} summarises the
packages which must be installed before \topspin\ can be used, and
provides a URL for each package. For each package, the version used
during the development of \topspin\ is specified.  Use of these or
newer versions is recommended.

\begin{figure}\begin{center}
\begin{small}
\begin{tabular}{|l|l|l|} \hline {\bf Package} & {\bf URL} & {\bf
Version}
\\\hline

Java runtime environment & \texttt{\javaurl} & \jreversion
\\\hline

\gap\ system & \texttt{\gapurl} & \gapversion \\\hline

 \spin\ model checker & \texttt{\spinurl} & \spinversion
\\\hline

 GNU C Compiler (gcc) & \texttt{\gccurl} &
\gccversion\\\hline

\end{tabular}\end{small}
\caption{\protect\topspin\
prerequisites.}\label{fig:symmextractorandtopspin:prerequisites}
\end{center}
\end{figure}

Before going further, make sure each of the packages of
\figref{symmextractorandtopspin:prerequisites} is installed on your
system

\vspace{2mm}
%
\noindent {\bf Important } Make sure that the \spin\ tool is
installed in such a way that it can be invoked by name from a
command prompt, i.e. so that typing \inline{spin} will launch \spin.
To verify that \spin\ is set up appropriately, type \inline{spin} in
a fresh command prompt.  You should see something like:
%
\begin{lstlisting}
C:\>spin Spin Version 5.1.6 -- 9 May 2008 reading input from stdin:
\end{lstlisting}
%
If you have renamed the \spin\ executable from \inline{spin} to
something like \inline{spin-linux} or \inline{spin516} then you will
get errors when you run \topspin.
%
\section{Downloading}\label{sec:downloadandinstall:downloading}

To download \topspin\ version \topspinversion, go to the \topspin\
web page:
%
\begin{lstlisting}
http://www.allydonaldson.co.uk/topspin/
\end{lstlisting}
%
and download the following file:
%
\begin{lstlisting}
TopSPIN_2.1.tgz
\end{lstlisting}
%
Use the \inline{tar} utility, the \emph{WinRAR}
tool,\footnote{\texttt{\winrarwebsite}} or another suitable program
to extract this archive to an appropriate location, e.g.
\inline{C:\\Program Files\\TopSPIN_2.1} under Windows, or
\inline{/usr/local/TopSPIN_2.1} under Linux.  This location is
referred to as the \topspin\ \emph{root directory}.

The archive should contain the following:

\begin{itemize}
\item {\bf \texttt{TopSPIN.jar} } Executable jar for the \topspin\ program
\item {\bf \texttt{documentation} } Folder containing this document, some related research papers
and a Ph.D. thesis
\item {\bf \texttt{examples} } Folder containing example Promela specifications for use with \topspin
\item {\bf \texttt{Common} } Folder containing various \gap\ and C files files used by \topspin
\item {\bf \texttt{saucy} } Folder containing source code for the \saucy\ program, which must be
compiled (as described in
\secref{downloadandinstall:installation:compilingsaucy}) before
\topspin\ can be used.
\end{itemize}

\section{Installing}\label{sec:downloadandinstall:installation}

\subsection{Compiling \saucy}\label{sec:downloadandinstall:installation:compilingsaucy}

\topspin\ uses a prototype extension of the \saucy\ program, which
is used to compute symmetries of directed graphs.  A version of
\saucy\ with this capability will eventually be available from the
\saucy\ website.\footnote{\texttt{\saucywebpage}} For the time
being, a source distribution of \saucy\ with the required extended
functionality is provided with \topspin.\footnote{Permission for
including the \saucy\ distribution with \topspin\ has been granted
by Paul Darga, lead developer of \saucy.}

Before using \topspin, you need to compile \saucy\ on your platform.
To do this, navigate to the \inline{saucy} folder, which is inside
the \topspin\ root directory, and type \inline{make}:
%
\begin{lstlisting}
C:\Program Files\TopSPIN_2.1>cd saucy C:\Program
Files\TopSPIN_2.1\saucy>make gcc -ansi -pedantic -Wall -O3   -c -o
main.o main.c gcc -ansi -pedantic -Wall -O3   -c -o saucy.o saucy.c
gcc -ansi -pedantic -Wall -O3   -c -o saucyio.o saucyio.c gcc  -o
saucy main.o saucy.o saucyio.o
\end{lstlisting}
%
\subsection{Creating a \protect\gap\ workspace}\label{sec:downloadandinstall:gapworkspace}
%
In order to start \gap\ efficiently, \topspin\ requires a \gap\
\emph{workspace} to be set up.  Essentially, a workspace is an image
of a \gap\ session with a selection of libraries and files already
loaded and ready to be executed.  For \topspin, the workspace
consists of the \gap\ components of \topspin\ which have been
developed for automatic symmetry reduction.  These are in the
\inline{Common} folder, inside the \topspin\ root directory.

Navigate into the \inline{Common} folder, and start \gap:
%
\begin{lstlisting}
C:\Program Files\TopSPIN_2.1>cd Common

C:\Program Files\TopSPIN_2.1\Common>gap

C:\Program Files\TopSPIN_2.1\Common>rem sample batch file for GAP

C:\Program Files\TopSPIN_2.1\Common>C:\GAP4R4\bin\gapw95.exe -m 14m
-l C:\GAP4R4\

            #########           ######         ###########           ###
         #############          ######         ############         ####
        ##############         ########        #############       #####
       ###############         ########        #####   ######      #####
      ######         #         #########       #####    #####     ######
     ######                   ##########       #####    #####    #######
     #####                    ##### ####       #####   ######   ########
     ####                    #####  #####      #############   ###  ####
     #####     #######       ####    ####      ###########    ####  ####
     #####     #######      #####    #####     ######        ####   ####
     #####     #######      #####    #####     #####         #############
      #####      #####     ################    #####         #############
      ######     #####     ################    #####         #############
      ################    ##################   #####                ####
       ###############    #####        #####   #####                ####
         #############    #####        #####   #####                ####
          #########      #####          #####  #####                ####

     Information at:  http://www.gap-system.org
     Try '?help' for help. See also  '?copyright' and  '?authors'

   Loading the library. Please be patient, this may take a while.
GAP4, Version: 4.4.6 of 02-Sep-2005, i686-pc-cygwin-gcc Components:
... Packages:    ... gap>
\end{lstlisting}
%
Now create a workspace as follows:
%
\begin{lstlisting}
gap> Read("WorkspaceGenerator.gap"); gap>
SaveWorkspace("gapworkspace"); true gap> quit;
\end{lstlisting}
%
Ensure that these commands are typed \emph{exactly} as shown,
ensuring that the semi-colon is included after the \inline{quit}
command.  When you exit \gap, you should find a file called
\inline{gapworksapce} in the \inline{Common} directory.
%
\section{Executing the \protect\topspin\ jar file}
%
It should now be possible to run the \topspin\ jar file on no input
to display a list of \topspin\ options:
%
\begin{lstlisting}
C:\>java -jar "C:\Program Files\TopSPIN_2.1\TopSPIN_2.1.jar"

Usage: topspin [check,detect] <filename>

Configuration file options:

 OPTION           PURPOSE           DEFAULT
 ------           -------           -------
  ...              ...               ...
\end{lstlisting}
%
The options displayed by the tool are described fully in
\chapref{overview}.  For now, successfully displaying these options
confirms that the Java runtime environment is correctly installed on
your system, and that the \topspin\ jar file you have obtained is in
working order.  If \topspin\ does \emph{not} correctly display its
options, proceed to \chapref{troubleshooting} to try to deduce what
is wrong, and to file a bug report if necessary.

In \chapref{example}, instructions for running \topspin\ to perform
symmetry reduction on a Promela specification are given.
