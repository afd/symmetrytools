\chapter{\workedexample}\label{chapter:example}

In this chapter, the basic workings of \topspin\ are illustrated via
a worked example.

\section{Loadbalancer specification}

The \inline{examples} folder in the \topspin\ root directory
contains a file, \inline{loadbalancer.p}.  This is a Promela
specification for a \emph{loadbalancer}, which forwards requests
from a pool of clients to a pool of servers in a fair manner.

Components in the loadbalancer are a set of 2 \emph{server} and 4
\emph{client} processes with associated communication channels, and
a \emph{loadbalancer} process with a dedicated input channel.  The
\emph{load} of a server is the number of messages queued on its
input channel.  Client processes send requests to the loadbalancer,
and if some server's channel is not full, the loadbalancer forwards
a request nondeterministically to one of the least-loaded server
queues. Each request contains a reference to the input channel of
its associated client process, and the server designated by the
loadbalancer uses this channel to service the request.

\section{Applying \protect\spin\ to the loadbalancer specification}\label{sec:example:applyingspin}

Before applying \topspin\ to this example, it is worth checking that
\spin\ is in good working order by model checking the example
without symmetry reduction.

Navigate to the \inline{examples} directory, and run the following
commands:
%
\begin{lstlisting}
C:\Program Files\TopSPIN_2.2\examples>spin -a loadbalancer.p
C:\Program Files\TopSPIN_2.2\examples>gcc -o pan -O2 -DSAFETY pan.c
C:\Program Files\TopSPIN_2.2\examples>pan -m100000
\end{lstlisting}
%
\spin\ should successfully verify that the model associated with the
specification is deadlock-free, producing output similar to:
%
\begin{lstlisting}
(Spin Version 5.1.6 -- 9 May 2008)
        + Partial Order Reduction

Full statespace search for:
        never claim             - (none specified)
        assertion violations    +
        cycle checks            - (disabled by -DSAFETY)
        invalid end states      +

State-vector 108 byte, depth reached 73413, errors: 0
   170903 states, stored
   413074 states, matched
   583977 transitions (= stored+matched)
        6 atomic steps
hash conflicts:     48755 (resolved)

   24.974       memory usage (Mbyte)

unreached in proctype client
        line 20, state 6, "-end-"
        (1 of 6 states)
unreached in proctype loadBalancer
        line 34, state 13, "-end-"
        (1 of 13 states)
unreached in proctype server
        line 43, state 7, "-end-"
        (1 of 7 states)
unreached in proctype :init:
        (0 of 9 states)

pan: elapsed time 0.818 seconds pan: rate 208927.87 states/second
\end{lstlisting}

\section{Setting up a \protect\topspin\ configuration file}\label{sec:example:configuration}

You are almost ready to use \topspin\ for symmetry reduction!  When
you invoke \topspin\ on a Promela specification, the tool requires
that a file called \inline{config.txt} is available in the current
directory.  This file tells \topspin\ where to find the \gap\ and
\saucy\ programs, the location of some common source code, and the
values of various runtime options. The file consists of a series of
lines, each of which has the form:

\vspace{2mm} \emph{attribute}\texttt{=}\emph{value} \vspace{2mm}

\noindent Three attributes are required in every configuration file:

\vspace{2mm}
\begin{enumerate}
\item {\bf \texttt{gap}} -- the absolute path for the shell script or batch file
required to launch \gap.  The value for
the \inline{gap} attribute must be such that typing this value at
the command prompt is all that is required to launch the \gap\
program.
\item {\bf \texttt{saucy}}  -- the absolute path for the \saucy\ executable.  Again, the
value for the \inline{saucy} attribute must be exactly what you type
to run \saucy\ from the command line.
\item {\bf \texttt{common}} -- the absolute path to the \inline{Common} directory in
the \topspin\ root directory, followed by a (back- or forward-,
depending on your operating system) slash.
\end{enumerate}
%
Under Windows, \inline{config.txt} might look like this:
%
\begin{lstlisting}
gap=C:\gap4r4\bin\gap.bat
saucy=C:\Program Files\TopSPIN_2.2\saucy\saucy.exe
common=C:\Program Files\TopSPIN_2.2\Common\
\end{lstlisting}
%
whereas a possible Linux version of \inline{config.txt} could be:
%
\begin{lstlisting}
gap=/home/username/bin/gap
saucy=/users/grad/ally/TopSPIN_2.2/saucy/saucy
common=/users/grad/ally/TopSPIN_2.2/Common/
\end{lstlisting}
%
Note that the \inline{gap} option should \emph{not} specify the
\gap\ executable itself, rather the shell script (\inline{gap.sh})
or batch file (\inline{gap.bat}) used to launch \gap.

The remainder of \inline{config.txt} is used to specify \topspin\
options for a particular specification.  One of these options is
discussed in \secref{example:enumerate}, and a complete overview of
options is given in \chapref{overview}.

A configuration file must always be present in the directory from
where you invoke \topspin.  You will probably use different
configuration files for different specifications, depending on the
nature of the symmetry associated with these specifications.
However, since the \inline{gap}, \inline{saucy} and \inline{common}
attributes are likely to be the same in all configuration files, it
makes sense to keep a ``skeleton'' configuration file containing
just these options, which you can then copy and extend for a given
Promela specification.

\section{Symmetry reduction with the \emph{fast} strategy}\label{sec:example:fast}

When you run \topspin, you can specify a symmetry reduction
\emph{strategy} for the tool to use.  The choice of strategy
influences the speed of symmetry reduction and the factor of
reduction obtained though the use of symmetry.  Some strategies
provide the maximum possible state-space reduction due to symmetry,
at the expense of a slow state-space search.  Other strategies are
more lightweight, providing potentially sub-optimal reduction, but
executing more quickly.

By default, \topspin\ uses the \emph{fast} strategy.  In a nutshell,
when using this strategy \topspin\ attempts to work out an efficient
symmetry reduction algorithm based on the type of symmetry
associated with the input specification.  The symmetry reduction
algorithm used is \emph{approximate} in the sense that it may result
in exploration of more than one state per symmetric equivalence
class.

\subsection{Running \protect\topspin}

Copy \inline{config.txt}, the basic configuration file created in
\secref{example:configuration}, into the \inline{examples} directory,
and run \topspin\ on \inline{loadbalancer.p} as
follows:\footnote{For reasons of space, some lines in the \topspin\
output have been wrapped. Also note that the output on your system
will vary slightly according to the paths you have specified in
\inline{config.txt}.}
%
\begin{lstlisting}
C:\Program Files\TopSPIN_2.2\examples>cp ../config.txt .
C:\Program Files\TopSPIN_2.2\examples>java -jar
          "C:\Program Files\TopSPIN_2.2\TopSPIN.jar" loadbalancer.p
File: loadbalancer.p
--------------------------------------
TopSPIN version 2.2
--------------------------------------
Configuration settings:
    Symmetry detection method: static channel diagram analysis
    Using 0 random conjugates
    Timeout for finding largest valid subgroup: 0 seconds
    Reduction strategy: FAST
    Using transpositions to represent permutations: true
    Using stabiliser chain for enumeration: true
    Using vectorisation: false
--------------------------------------

Typechecking input specification...

Specification is well typed!

Launching saucy via the following command: C:\Program Files\TopSPIN_2.2\saucy\saucy.exe -d
   "C:\Program Files\TopSPIN_2.2\Common\graph.saucy"

Starting GAP with command: C:\gap4r4\bin\gap.bat -L
   "C:\Program Files\TopSPIN_2.2\Common\gapworkspace" -q

The group:
   G = <(3 4)(clients2 clients1),(3 2)(clients0 clients1),(servers1 servers0)(7 6),
        (5 4)(clients2 clients3)>
is a valid group for symmetry reduction.

Generating symmetry reduction algorithms

The symmetry group has size 48
Completed generation of sympan verifier which includes algorithms for symmetry reduction!

To generate an executable verifier use the following command:
   gcc -o sympan sympan.c group.c
together with SPIN compile-time directives for your specification.

Execute the verifier using the following command:
   sympan.exe
together with SPIN run-time options for your specification.
\end{lstlisting}
%
If \topspin\ does \emph{not} produce output like this, then proceed
to \chapref{troubleshooting} to try to deduce what is wrong, and to
file a bug report if necessary.

The first part of the \topspin\ output specifies which version of
the tool is being used, and how the various tool options have been
configured. These options have all been set to default values.  For
now, simply note that the \emph{fast} strategy is being used.

The tool then typechecks the input specification, reporting that the
specification is well-typed.

After typechecking, the \saucy\ and \gap\ programs are launched, to
perform automatic symmetry detection.  Note that the paths to these
tools have been taken from the configuration file.  The \saucy\ tool
is passed the \inline{-d} option to indicate that the graph it will
operate on is \emph{directed}.  A temporary file,
\inline{graph.saucy}, is also passed to the program: this is a graph
generated by \topspin\ from which symmetries of
\inline{loadbalancer.p} are derived.  The \gap\ package is launched
with the \inline{-L} option to indicate that a workspace
(\secref{downloadandinstall:gapworkspace}) should be loaded.  The
path to this workspace -- \inline{gapworkspace} inside the
\inline{Common} directory -- is also passed to \gap.  The
\inline{-q} operation launches \gap\ in \emph{quiet} mode, which
improves the efficiency of communication between the \gap\ and Java
components of \topspin.

\topspin\ successfully computes generators for a group of symmetries
associated with the loadbalancer specification.  From the generators
of this group, it can be seen that there is full symmetry between
the \emph{client} processes, as well as symmetry between the
\emph{server} processes.  \topspin\ then goes on to generate
symmetry reduction algorithms for this group, reporting that the
number of symmetries is 48 (4! symmetries between the clients, and 2
symmetries between the servers, leads to $24\times 2=48$ symmetries
overall).

The end of the \topspin\ output details how the $C$ files generated
by \spin\ and \topspin\ should be compiled and executed.

\subsection{Compiling and executing the {\bf \texttt{sympan}} verifier}

The \inline{examples} directory should now contain a number of
files, including \inline{sympan.c} and \inline{group.c}.  The
\inline{sympan.c} file is a modified version of the \inline{pan.c}
file produced by \spin, which includes symmetry reduction
algorithms.  The \inline{group.c} file includes functions for
manipulating permutations.

Compile and execute this verifier using commands analogous to those
shown in \secref{example:applyingspin}:
%
\begin{lstlisting}
C:\Program Files\TopSPIN_2.2\examples>gcc -o sympan -O2 -DSAFETY sympan.c group.c
C:\Program Files\TopSPIN_2.2\examples>sympan -m100000

(Spin Version 5.1.6 -- 9 May 2008)
        + Partial Order Reduction

Full statespace search for:
        never claim             - (none specified)
        assertion violations    +
        cycle checks            - (disabled by -DSAFETY)
        invalid end states      +

State-vector 108 byte, depth reached 2323, errors: 0
     4960 states, stored
    12254 states, matched
    17214 transitions (= stored+matched)
        6 atomic steps
hash conflicts:        35 (resolved)

    5.735       memory usage (Mbyte)

unreached in proctype client
        line 20, state 6, "-end-"
        (1 of 6 states)
unreached in proctype loadBalancer
        line 34, state 13, "-end-"
        (1 of 13 states)
unreached in proctype server
        line 43, state 7, "-end-"
        (1 of 7 states)
unreached in proctype :init:
        (0 of 9 states)

pan: elapsed time 0.369 seconds pan: rate 13441.734 states/second
\end{lstlisting}
%
Comparing this model checking result with the result without
symmetry reduction (\secref{example:applyingspin}) shows that the
\emph{fast} strategy leads to a state-space reduction factor of
34.5.  The theoretical maximum symmetry reduction factor is 48,
which is the size of the symmetry group computed by \topspin, so
this is a reasonably good result.  Notice that speedup factor is
just 2.2, and as a result the \emph{states/second} rate for
verification is significantly lower when symmetry reduction is
applied.
%
\section{Symmetry reduction with the \emph{enumerate} strategy}\label{sec:example:enumerate}
%
Although the \emph{fast} strategy provides effective symmetry
reduction for the loadbalancer example, the strategy does not
provide space-optimal symmetry reduction. The \emph{enumerate}
strategy, on the other hand, is guaranteed to provide full symmetry
reduction.

To use the \emph{enumerate} strategy, open \inline{config.txt} in
the \inline{examples} directory, and add the following line:
%
\begin{lstlisting}
strategy=enumerate
\end{lstlisting}
%
This additional option tells \topspin\ to use the \emph{enumerate}
strategy over the default \emph{fast} strategy.

Now apply \topspin\ to the loadbalancer specification, and compile
and run the generated verifier:
%
\begin{lstlisting}
C:\Program Files\TopSPIN_2.2\examples>java -jar
"C:\Program Files\TopSPIN_2.2\TopSPIN.jar" loadbalancer.p

   ... TopSPIN output ...

C:\Program Files\TopSPIN_2.2\examples>gcc -o sympan -O2 -DSAFETY sympan.c group.c
C:\Program Files\TopSPIN_2.2\examples>sympan -m100000

(Spin Version 5.1.6 -- 9 May 2008)
        + Partial Order Reduction

Full statespace search for:
        never claim             - (none specified)
        assertion violations    +
        cycle checks            - (disabled by -DSAFETY)
        invalid end states      +

State-vector 108 byte, depth reached 1916, errors: 0
     4213 states, stored
    11181 states, matched
    15394 transitions (= stored+matched)
        6 atomic steps
hash conflicts:        23 (resolved)

    5.638       memory usage (Mbyte)

unreached in proctype client
        line 20, state 6, "-end-"
        (1 of 6 states)
unreached in proctype loadBalancer
        line 34, state 13, "-end-"
        (1 of 13 states)
unreached in proctype server
        line 43, state 7, "-end-"
        (1 of 7 states)
unreached in proctype :init:
        (0 of 9 states)

pan: elapsed time 0.647 seconds pan: rate  6511.592 states/second
\end{lstlisting}
%
Verification using the \emph{enumerate} strategy provides a better
reduction factor: 40.6 vs. 34.5 with the \emph{fast} strategy
(\secref{example:fast}). This comes at an expense: the speedup
factor is reduced from 2.2 for the \emph{fast} strategy to 1.3.  For
specifications with more symmetric components, this time penalty is
more significant: the \emph{enumerate} strategy works by applying
\emph{every} symmetry associated with the input specification to
\emph{every} state encountered during model checking.  Since a
specification with $n$ symmetric components has a symmetry group of
size at least $n!$, use of the \emph{enumerate} strategy quickly
becomes infeasible.

\section{Summary so far}
%
You should now have successfully installed \topspin\ and its
prerequisite components, created a configuration file, and applied
the tool to a Promela example.  If you have encountered any problems
during this process then proceed to \chapref{troubleshooting} for
troubleshooting ideas and information on how to report \topspin\
bugs.

At this stage, you may have learned all you need to know about
\topspin\ for your basic symmetry reduction needs.
\chapsref{overview}{compilingfromsource} are for advanced users who
wish to explore \topspin's more sophisticated options, and compile
the tool from source, respectively.
